\documentclass[usenames,dvipsnames,aspectratio=169,11pt,envcountsect]{beamer}
%handout,
%aspectratio=43

\usepackage{others/prespreamble}
\usepackage{natbib}
\usepackage{transparent}
\newcommand{\fade}[1]{\texttransparent{0.15}{#1}} % 15% opaque (very faint)
\usepackage[normalem]{ulem}
\usepackage{colortbl, xcolor}
\definecolor{RoyalBlue}{rgb}{0.19, 0.55, 0.91}
\linespread{1}
\usepackage{mathrsfs}
\usepackage{transparent}

\usefonttheme{serif}
\usefonttheme{professionalfonts}
\usepackage{lmodern}
\usepackage{csquotes}

\begin{document}

% ---- Title slide -----------------------------
\begin{frame}[noframenumbering,plain]
	\maketitle

\end{frame}

% ==== 1. Motivation ======================================================================
\begin{frame}\frametitle{What is game theory?}

	A logic to think about interacting individuals. \pause

	\vfill

	\begin{enumerate}
		\item \textbf{Everyday life}: brothers and sisters deciding how to split their toys. \pause
		      \vfill
		\item \textbf{Economics and business}:
		      \begin{enumerate}
			      \item firms setting prices in competition with each other;
			      \item manager and employee negotiating a salary.
		      \end{enumerate} \pause
		      \vfill
		\item \textbf{Political science}:
		      \begin{enumerate}
			      \item political candidates choosing their electoral platforms;
			      \item countries negotiating trade agreements.
		      \end{enumerate} \pause
		      \vfill
		\item \textbf{Biology and evolutionary theory}: cancer cells competing with other cells. \pause
		      \vfill
		\item \textbf{Computer science}: software agents bidding in online markets or auctions.

	\end{enumerate}


\end{frame}

\begin{frame}
	\frametitle{A bit of history}

	\begin{columns}[c]  % vertically centered
		% ---- Left: picture of von Neumann ----
		\begin{column}{0.35\textwidth}
			\centering
			\includegraphics[width=\linewidth]{others/Figures/vonneumann.jpg}% replace path/name
		\end{column}

		% ---- Right: text ----
		\begin{column}{0.65\textwidth}
			The birth of modern game theory is usually traced back to the publication of \textit{Theory of Games and Economic Behavior} by \citeauthor{vonneumannTheoryGamesEconomic2007} in 1944. \pause

			\vspace{1cm}

			Later, von Neumann became involved in the development of nuclear weapons as part of the Manhattan Project, and with the US military in general. \pause

			\vspace{1cm}

			In the 60--70's it started permeating into economics, and later into other fields.
		\end{column}
	\end{columns}

\end{frame}


\begin{frame}
	\frametitle{Program for today}

	Discuss a simple example of a game: \textbf{the Prisoner's Dilemma}. \pause

	\vfill

	Observe that the prisoner's dilemma helps understanding competition in markets. \pause

	\vfill

	The game-theoretic study helps us developing policies to improve collective welfare.

\end{frame}

\begin{frame}<1-5>
	\frametitle{A simple example}

	Ann has an apple, Bob has a banana.

	\vfill

	However, Ann prefers bananas over apples, and Bob prefers apples over bananas.

	\vfill

	They can either choose to \textbf{Give} their fruit to the other person, or \textbf{Keep} it.

	\vfill

	\begin{columns}[c]
		% -------- LEFT: payoff table --------
		\begin{column}{0.5\textwidth}
			\centering
			\begin{table}[h]
				\centering
				\begin{tabular}{c|c|c|}
					             & \multicolumn{2}{c}{\textbf{Bob}}                                     \\
					\cline{2-3}
					\textbf{Ann} & Give                                                          & Keep \\
					\hline
					Give
					             & \onslide<2->{\alt<2>{\textcolor{blue}{\(6,\,6\)}}{\(6,\,6\)}}        % (Give,Give)
					             & \onslide<3->{\alt<3>{\textcolor{blue}{\(2,\,8\)}}{\(2,\,8\)}}        % (Give,Keep)
					\\
					\hline
					Keep
					             & \onslide<4->{\alt<4>{\textcolor{blue}{\(8,\,2\)}}{\(8,\,2\)}}        % (Keep,Give)
					             & \onslide<5->{\alt<5>{\textcolor{blue}{\(4,\,4\)}}{\(4,\,4\)}}        % (Keep,Keep)
					\\
					\hline
				\end{tabular}
			\end{table}
		\end{column}

		% -------- RIGHT: illustration --------
		\begin{column}{0.3\textwidth}
			\centering
			\includegraphics[width=0.9\linewidth]{others/Figures/ann-bob.png}%
			% replace with the actual path/name of your image file
		\end{column}
	\end{columns}

\end{frame}

\begin{frame}<1-4>
	\frametitle{What should Ann do?}

	\begin{table}[h]
		\centering
		\begin{tabular}{c|c|c|}
			 & \multicolumn{2}{c}{\textbf{Bob}}  \\
			\cline{2-3}
			\textbf{Ann}
			 &                                   % Bob: Give
			\alt<3>{\fade{Give}}{Give}
			 &                                   % Bob: Keep
			\alt<2>{\fade{Keep}}{Keep}           \\
			\hline
			% ---------- Ann: Give row ----------
			Give
			 &                                   % (Give,Give): faded only when Bob=Keep fixed (slide 3)
			\alt<3>{\fade{\(6,\,6\)}}{\(6,\,6\)}
			 &                                   % (Give,Keep): faded only when Bob=Give fixed (slide 2)
			\alt<2>{\fade{\(2,\,8\)}}{\(2,\,8\)} \\
			\hline
			% ---------- Ann: Keep row ----------
			\textcolor<2-4>{red}{Keep}
			 &                                   % (Keep,Give): Ann's payoff 8 red on 2 & 4; faded on 3
			\alt<3>{\fade{\(8,\,2\)}}{%
				\(\textcolor<2,4>{red}{8},\,2\)%
			}
			 &                                   % (Keep,Keep): Ann's payoff 4 red on 3 & 4; faded on 2
			\alt<2>{\fade{\(4,\,4\)}}{%
				\(\textcolor<3-4>{red}{4},\,4\)%
			}                                    \\
			\hline
		\end{tabular}
	\end{table}
\end{frame}

\begin{frame}<1-3>
	\frametitle{What should Bob do?}

	\begin{table}[h]
		\centering
		\begin{tabular}{c|c|c|}
			 & \multicolumn{2}{c}{\textbf{Bob}} \\
			\cline{2-3}
			\textbf{Ann}
			 & Give
			 & \textcolor<2-3>{red}{Keep}       \\   % Bob's best-reply action
			\hline
			% ---------- Ann: Give row ----------
			\alt<3>{\fade{Give}}{Give}
			 &                                  % (Give,Give): faded only when Ann=Keep fixed (3)
			\alt<3>{\fade{\(6,\,6\)}}{\(6,\,6\)}
			 &                                  % (Give,Keep): Bob's payoff 8 red on 2; faded on 3
			\alt<3>{\fade{\(2,\,8\)}}{%
				\(2,\,\textcolor<2>{red}{8}\)%
			}                                   \\
			\hline
			% ---------- Ann: Keep row ----------
			\alt<2>{\fade{Keep}}{Keep}
			 &                                  % (Keep,Give): faded only when Ann=Give fixed (2)
			\alt<2>{\fade{\(8,\,2\)}}{\(8,\,2\)}
			 &                                  % (Keep,Keep): Bob's payoff 4 red on 3
			\alt<2>{\fade{\(4,\,4\)}}{%
				\(4,\,\textcolor<3>{red}{4}\)%
			}                                   \\
			\hline
		\end{tabular}
	\end{table}
\end{frame}

\begin{frame}
	\frametitle{What happens then?}

	\begin{table}[h]
		\centering
		\begin{tabular}{c|c|c|}
			 & \multicolumn{2}{c}{\textbf{Bob}} \\
			\cline{2-3}
			\textbf{Ann}
			 & Give
			 & \textcolor{red}{Keep}            \\
			\hline
			Give
			 & \(6,\,6\)
			 & \(2,\,8\)                        \\
			\hline
			\textcolor{red}{Keep}
			 & \(8,\,2\)
			 & \(\textcolor{red}{4,\,4}\)       \\
			\hline
		\end{tabular}
	\end{table}
\end{frame}

\begin{frame}<1-5>
	\frametitle{Innovate or not?}

	Two firms produce plastic bottles.

	\vfill

	Each firm can either \textbf{Innovate} (invest in new technology to make biodegradable bottles) or \textbf{Not} (keep producing regular plastic bottles).

	\vfill

	Innovating is costly, and makes the bottle more expensive.

	\vfill

	\begin{columns}[c]
		% ---------- LEFT: payoff table ----------
		\begin{column}{0.5\textwidth}
			\centering
			\begin{table}[h]
				\centering
				\begin{tabular}{c|c|c|}
					                  & \multicolumn{2}{c}{\textbf{Column Firm}}                            \\
					\cline{2-3}
					\textbf{Row Firm} & Innovate                                                      & Not \\
					\hline
					Innovate
					                  & \onslide<2->{\alt<2>{\textcolor{blue}{\(6,\,6\)}}{\(6,\,6\)}}       % (Innovate,Innovate)
					                  & \onslide<3->{\alt<3>{\textcolor{blue}{\(2,\,8\)}}{\(2,\,8\)}}       % (Innovate,Not)
					\\
					\hline
					Not
					                  & \onslide<4->{\alt<4>{\textcolor{blue}{\(8,\,2\)}}{\(8,\,2\)}}       % (Not,Innovate)
					                  & \onslide<5->{\alt<5>{\textcolor{blue}{\(4,\,4\)}}{\(4,\,4\)}}       % (Not,Not)
					\\
					\hline
				\end{tabular}
			\end{table}
		\end{column}

		% ---------- RIGHT: illustration ----------
		\begin{column}{0.3\textwidth}
			\centering
			\includegraphics[width=0.9\linewidth]{others/Figures/firms.png}%
			% replace with the actual path/name of your factories image
		\end{column}
	\end{columns}

\end{frame}


\begin{frame}
	\frametitle{What happens?}

	The strategic interaction in the market leads to a lack of innovation.

	\vfill

	Even if everyone would be better off if both firms innovated!

	\vfill

	\begin{table}[h]
		\centering
		\begin{tabular}{c|c|c|}
			 & \multicolumn{2}{c}{\textbf{Column Firm}} \\
			\cline{2-3}
			\textbf{Row Firm}
			 & Innovate
			 & \textcolor{red}{Not}                     \\
			\hline
			Innovate
			 & \(6,\,6\)
			 & \(2,\,8\)                                \\
			\hline
			\textcolor{red}{Not}
			 & \(8,\,2\)
			 & \(\textcolor{red}{4,\,4}\)               \\
			\hline
		\end{tabular}
	\end{table}
\end{frame}

\begin{frame}<1-2>
	\frametitle{What can be done?}

	Obliging firms to do something is usually not a good idea.

	\vfill

	There are advantages in letting firms decide by themselves.

	\vfill

	But we can try to change the \textbf{incentives} they face.

	\vfill

	\textbf{Taxing pollution} is a common policy tool.

	\vfill

	\begin{table}[h]
		\centering
		\begin{tabular}{c|c|c|}
			                  & \multicolumn{2}{c}{\textbf{Column Firm}}         \\
			\cline{2-3}
			\textbf{Row Firm} & Innovate                                   & Not \\
			\hline
			Innovate
			                  & \(6,\,6\)
			                  & \(2,\,8\onslide<2->{\textcolor{red}{-3}}\)       \\
			\hline
			Not
			                  & \(8\onslide<2->{\textcolor{red}{-3}},\,2\)
			                  & \(4\onslide<2->{\textcolor{red}{-3}},\,
			4\onslide<2->{\textcolor{red}{-3}}\)                                 \\
			\hline
		\end{tabular}
		% taxes appear on <2>
	\end{table}
\end{frame}

\begin{frame}
	\frametitle{A new game}

	After introducing taxes, the game looks like this.

	\vfill

	\begin{table}[h]
		\centering
		\begin{tabular}{c|c|c|}
			                  & \multicolumn{2}{c}{\textbf{Column Firm}}              \\
			\cline{2-3}
			\textbf{Row Firm} & Innovate                                 & Not        \\
			\hline
			Innovate          & \(6,\, 6\)                               & \(2,\, 5\) \\
			\hline
			Not               & \(5,\, 2\)                               & \(1,\, 1\) \\
			\hline
		\end{tabular}
		% \caption{Prisoner's Dilemma with taxes: +1 for Cooperate, -3 for Defect.}
	\end{table}

\end{frame}

\begin{frame}<1-4>
	\frametitle{What does row firm do?}

	\begin{table}[h]
		\centering
		\begin{tabular}{c|c|c|}
			 & \multicolumn{2}{c}{\textbf{Column Firm}} \\
			\cline{2-3}
			\textbf{Row Firm}
			 &                                          % Column: Innovate
			\alt<3>{\fade{Innovate}}{Innovate}
			 &                                          % Column: Not
			\alt<2>{\fade{Not}}{Not}                    \\
			\hline
			% ---------- Row: Innovate ----------
			\textcolor<2-4>{red}{Innovate}
			 &                                          % (Innovate, Innovate)
			\alt<3>{\fade{\(6,\,6\)}}{%
				\(\textcolor<2,4>{red}{6},\,6\)%
			}
			 &                                          % (Innovate, Not)
			\alt<2>{\fade{\(2,\,5\)}}{%
				\(\textcolor<3-4>{red}{2},\,5\)%
			}                                           \\
			\hline
			% ---------- Row: Not ----------
			Not
			 &                                          % (Not, Innovate)
			\alt<3>{\fade{\(5,\,2\)}}{\(5,\,2\)}
			 &                                          % (Not, Not)
			\alt<2>{\fade{\(1,\,1\)}}{\(1,\,1\)}        \\
			\hline
		\end{tabular}
	\end{table}
\end{frame}

\begin{frame}<1-3>
	\frametitle{What does column firm do?}

	\begin{table}[h]
		\centering
		\begin{tabular}{c|c|c|}
			 & \multicolumn{2}{c}{\textbf{Column Firm}} \\
			\cline{2-3}
			\textbf{Row Firm}
			 & \textcolor<2-3>{red}{Innovate}
			 & Not                                      \\   % Column's best-reply action is Innovate
			\hline
			% ---------- Row: Innovate ----------
			\alt<3>{\fade{Innovate}}{Innovate}
			 &                                          % (Innovate, Innovate)
			\alt<3>{\fade{\(6,\,6\)}}{%
				\(6,\,\textcolor<2>{red}{6}\)%
			}
			 &                                          % (Innovate, Not)
			\alt<3>{\fade{\(2,\,5\)}}{\(2,\,5\)}        \\
			\hline
			% ---------- Row: Not ----------
			\alt<2>{\fade{Not}}{Not}
			 &                                          % (Not, Innovate)
			\alt<2>{\fade{\(5,\,2\)}}{%
				\(5,\,\textcolor<3>{red}{2}\)%
			}
			 &                                          % (Not, Not)
			\alt<2>{\fade{\(1,\,1\)}}{\(1,\,1\)}        \\
			\hline
		\end{tabular}
	\end{table}
\end{frame}

\begin{frame}
	\frametitle{What happens now?}

	\begin{table}[h]
		\centering
		\begin{tabular}{c|c|c|}
			 & \multicolumn{2}{c}{\textbf{Column Firm}} \\
			\cline{2-3}
			\textbf{Row Firm}
			 & \textcolor{red}{Innovate}
			 & Not                                      \\
			\hline
			\textcolor{red}{Innovate}
			 & \(\textcolor{red}{6,\,6}\)
			 & \(2,\,5\)                                \\
			\hline
			Not
			 & \(5,\,2\)
			 & \(1,\,1\)                                \\
			\hline
		\end{tabular}
	\end{table}
\end{frame}

\begin{frame}
	\frametitle{What do people vote?}

	Recently, researchers studied what game people vote to play in experiments.

	\vfill

	\begin{columns}[c]
		% ---- Left column: old game ----
		\begin{column}{0.48\textwidth}
			\centering
			\textbf{Before taxes and subsidies}\\[0.5em]
			\begin{tabular}{c|c|c|}
				         & \multicolumn{2}{c}{\textbf{Column Firm}}                              \\
				\cline{2-3}
				\textbf{Row Firm}
				         & Innovate                                 & Not                        \\
				\hline
				Innovate & \(6,\,6\)                                & \(3,\,8\)                  \\
				\hline
				Not      & \(8,\,2\)                                & \(\textcolor{red}{4,\,4}\) \\
				\hline
			\end{tabular}
		\end{column}

		% ---- Right column: new game ----
		\begin{column}{0.48\textwidth}
			\centering
			\textbf{After taxes and subsidies}\\[0.5em]
			\begin{tabular}{c|c|c|}
				         & \multicolumn{2}{c}{\textbf{Column Firm}}             \\
				\cline{2-3}
				\textbf{Row Firm}
				         & Innovate                                 & Not       \\
				\hline
				Innovate & \(\textcolor{red}{6,\,6}\)               & \(4,\,5\) \\
				\hline
				Not      & \(5,\,4\)                                & \(1,\,1\) \\
				\hline
			\end{tabular}
		\end{column}
	\end{columns}

	\vfill

	A lot of people vote to play the game before taxes and subsidies!

	\vfill

	They are not able to predict the effect of changing incentives.

\end{frame}



\begin{frame}
	\frametitle{Summary}

	Game theory provides a framework to study strategic interactions.

	\vfill

	It can help understanding real-world phenomena, and design better policies.

	\vfill

	Today we saw a simple example: the Prisoner's Dilemma.

	\vfill

	There are many more concepts and tools in game theory to explore!
\end{frame}


\begin{frame}[noframenumbering,plain]

	\frametitle{References}

	%\nocite{*}
	\bibliography{chapters/references}
	\bibliographystyle{apacite}

\end{frame}

\end{document}