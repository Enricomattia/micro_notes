
\documentclass[usenames,dvipsnames,aspectratio=169,11pt,handout]{beamer}
%handout,
%aspectratio=43

\usepackage{others/discpreamble}
\usepackage{natbib}
\usepackage[normalem]{ulem}
\usepackage{colortbl, xcolor}
\definecolor{RoyalBlue}{rgb}{0.19, 0.55, 0.91}
\linespread{1}
\usepackage{mathrsfs}
\usepackage{csquotes}

% small bibliography
\let\oldthebibliography=\thebibliography
\renewcommand{\thebibliography}[1]{
    \oldthebibliography{#1}
    \setlength{\itemsep}{2pt}
    \tiny
}

\begin{document}

\begin{frame}[noframenumbering,plain]
	\maketitle
\end{frame}

\begin{frame}
	\frametitle{Epistemic game theory in 30 seconds}

	Game theory = individual preferences (tastes and beliefs) + equilibrium analysis.

	\vfill

	Equilibrium = rationality (expected utility) + common belief in rationality.

	\vfill

	Behavioural economics introduces “behavioural heuristics” into games.

	\vfill

	How these heuristics relate to classical preferences + equilibrium is often unclear.

	\vfill

	Epistemic game theory takes these sums \underline{\textbf{very seriously}}.
\end{frame}

\begin{frame}\frametitle{Motivation of this paper}

	Dynamic refinements (e.g. subgame perfection) use the chronological structure of games.

	\vfill

	Refinements are equivalent to having specific beliefs about others’ rationality.

	\vfill

	Classical refinements often mispredict behaviour in classic experiments.

	\vfill

	Question: how robust are classical dynamic refinements to small doubts about rationality?
\end{frame}

\begin{frame}\frametitle{Framework and concepts}

	Finite multistage games with observable actions.

	\vfill

	Players hold beliefs over others’ strategies.

	\vfill

	Dynamic rationalizability: weak (initial), backward (induction), strong (forward induction).

	\vfill

	Introduce \(p\)-belief: an event is believed with probability at least \(p<1\). \citep{mondererApproximatingCommonKnowledge1989}

	\vfill

	Define weak, backward, strong \(p\)-rationalizability.

\end{frame}

\begin{frame}\frametitle{Main results: rationalizability}

	For \(p=1\): \(p\)-versions coincide with classical notions.

	\vfill

	For any \(p<1\):
	strong and backward \(p\)-rationalizability \textbf{collapse to weak (initial) rationalizability}.

	\vfill

	Interpretation:
	even tiny doubts about higher-order rationality destroy the refinement power of dynamic reasoning.

\end{frame}

\begin{frame}\frametitle{Behavioural implications}

	With \(p<1\), dynamic refinements lose bite; predictions become essentially static.

	\vfill

	Centipede game:
	cooperation in all but the last stage is \(p\)-rationalisable.

	\vfill

	Finitely repeated Prisoner’s Dilemma:
	all threshold strategies except “cooperate in the final round” are \(p\)-rationalisable.

	\vfill

	Small doubts about rationality can rationalise non-inductive play in these games.

	\vfill

	\textbf{But I add}: for any value of \( p \) you could probably rationalise many things!

\end{frame}

\begin{frame}\frametitle{Comments}

	Results induce a conceptual dilemma:

	\vfill

	\begin{itemize}
		\item Better dynamic refinements we did not develop yet? \citep{siniscalchiStructuralRationalityDynamic2022}
		\item Is the notion of \( p \)-belief in this paper inadequate?
		\item (\textbf{Pessimistic}): there is no hope.
	\end{itemize}

	\vfill

	About the definition of \( p \)-belief:
	it is quite weak.

	\vfill

	If you impose \(p\)-belief at each node you have a tension with Bayesian updating.

	\vfill

	Anything in between?

	\begin{itemize}
		\item Only on first order beliefs;
		\item Only locally but not on continuation play;
		\item Might choose how to solve trade-off with Bayesian updating.
	\end{itemize}

\end{frame}

\begin{frame}[noframenumbering,plain]

	\frametitle{References}

	%\nocite{*}
	\bibliography{chapters/references}
	\bibliographystyle{apacite}

\end{frame}

\end{document}