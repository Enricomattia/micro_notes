\renewcommand{\thefootnote}{\fnsymbol{footnote}}

\chapter[Existence of competitive equilibria]%
 {Existence of competitive equilibria}%
\label{ch:L10}

% 3) Reset things so later footnotes go back to 1, 2, 3, …
%\setcounter{footnote}{0}
\renewcommand{\thefootnote}{\arabic{footnote}}

We assume preferences are continuous, strictly convex, and strongly monotone. Under these assumptions, we can prove that a competitive equilibrium exists.

\begin{definition}
    A preference relation is \textbf{strongly monotone} if for all \(x, y \in \mathbb{R}^L_+\) such that \(x \geq y\) and \(x \neq y\), we have \(x \succ y\).
\end{definition}

\begin{definition}
    A preference relation is \textbf{strictly convex} if for all \(x, y, z \in \mathbb{R}^L_+\) such that \(y \succ x\) and \(z \succ x\), we have for all \(\alpha \in (0,1)\), \(\alpha y + (1-\alpha) z \succ x\). Equivalently, all upper contour sets are strictly convex sets.
\end{definition}

If a preference relation is continuous, strictly convex, and strongly monotone, then the Walrasian demand is single-valued and can therefore be viewed as a function. We can then define the excess demand function as follows.

\begin{definition}
    The \textbf{excess demand function} of and individual \( i \) with a single-valued Walrasian demand function \( D_i(p, e_i) \) is given by
    \[z_i(p, e_i) = D_i(p, e_i) - e_i.\]
\end{definition}

From the individual excess demand functions, we can construct the aggregate excess demand function.

\[
    z(p) = \sum_i z_i (p) .
\]

The excess demand function maps prices to allocations. Under our assumptions on preferences, Walrasian equilibrium can be characterised through the excess demand function as follows.

\begin{proposition}\label{prop:walr_excess}
    If individual preferences \( \succsim_i \) are continuous, strictly convex, and strongly monotone for each \( i \), then an allocation \( x \) is a Walrasian equilibrium if and only if there exists a price vector \( p \) such that \( z (p) = 0 \).
\end{proposition}

The excess demand function has several important properties that we will use to prove the existence of a competitive equilibrium.

\begin{proposition}\label{prop:excess}
    If individual preferences \( \succsim_i \) are continuous, strictly convex, and strongly monotone for each \( i \), then the aggregate excess demand function \( z(p) \) satisfies the following properties:
    \begin{enumerate}
        \item \( z(p) \) is \textbf{homogeneous of degree zero}:\( z( \alpha p) = z(p) \) for all \( \alpha > 0 \);
        \item \( z(p) \) satisfies Walras' law: \( p \cdot z(p) = 0 \) for all strictly positive prices \( p \);
        \item \( z(p) \) is continuous;
        \item there is an \( s > 0 \) such that for all goods \( \ell \) and prices \( p \), \( z_{\ell} (p) > -s \);
        \item if \(p^n \) is a sequence of prices converging to \( p \) with \( p_{\ell} = 0 \) for some good \( \ell \), then \( \max \{ z^1 (p^n), \ldots, z^L (p^n) \} \to +\infty \).
    \end{enumerate}
\end{proposition}

\begin{theorem}\label{thm:kaku} (\textbf{Kakutani's fixed point})
    Let \( A \subseteq \mathbb{R}^L \) be a non-empty, compact, and convex set, and \( f : A \rightrightarrows A \) be an upper hemicontinuous correspondence such that \( f(x) \subseteq A \) is non-empty and convex for each \( x \in A \). Then, \( f \) has a fixed point, i.e., there exists \( x \in A \) such that \( x \in f(x) \).
\end{theorem}

\begin{proposition}
    If the aggregate excess demand function \( z(p) \) satisfies the properties in Proposition \ref{prop:excess}, then there exists a price vector \( p \) such that \( z(p) = 0 \). Therefore, in such economy a Walrasian equilibrium exists.
\end{proposition}

\begin{proof}
    Because \( z(\cdot) \) is homogeneous of degree zero we can restrict attention to prices on the \emph{unit simplex}

    \[
        \Delta
        :=
        \Bigl\{
        p \in \mathbb{R}^\ell_+ \;\Bigm|\;
        \sum_{\ell=1}^\ell p^\ell = 1
        \Bigr\}.
    \]

    Let

    \[
        \text{Interior }\Delta
        :=
        \{ p \in \Delta : p^\ell > 0 \text{ for all } \ell \}
        \quad\text{and}\quad
        \text{Boundary }\Delta := \Delta \setminus \text{Interior }\Delta .
    \]

    Recall that \( z(\cdot) \) was originally defined for strictly positive
    prices only; by continuity, we can extend it uniquely to \( \Delta \).

    We construct a correspondence
    \( f : \Delta \rightrightarrows \Delta \) and then apply Kakutani's
    fixed–point theorem to \( f \). For notational simplicity, whenever
    \( q \in f(p) \) we write just ``\( q \)'' for such a vector.

    \paragraph{Step 1. Construction on Interior \( \Delta \).}
    For \( p \in \text{Interior }\Delta \) and \( p>0 \), define
    \[
        f(p)
        :=
        \bigl\{
        q \in \Delta
        \;\bigm|\;
        z(p)\cdot q \ge z(p)\cdot q'
        \text{ for all } q' \in \Delta
        \bigr\}.
    \]
    In words, given the current ``proposal'' \( p \), the correspondence
    \( f(\cdot) \) selects price vectors \( q \) that, among all admissible price
    vectors on the simplex, maximise the value of the (aggregate) excess
    demand vector \( z(p) \).

    Write \( z^\ell(p) \) for the excess demand of good \( \ell \) at prices \( p \).
    From the definition of \( f(p) \) one easily checks that
    \[
        f(p)
        =
        \Bigl\{
        q \in \Delta
        \;\Bigm|\;
        q^\ell = 0
        \text{ whenever }
        z^\ell(p) < \max\{ z^1(p),\dots,z^\ell(p)\}
        \Bigr\}.
    \]
    Indeed, if \( q \) put positive weight on some good with strictly lower
    excess demand than the maximum, we could reallocate that weight
    towards a good with maximal excess demand and increase
    \( z(p)\cdot q \), contradicting optimality of \( q \).

    By Walras' law, \( p \cdot z(p) = 0 \) for all strictly positive \( p \).
    Hence if \( z(p)\neq 0 \), at least one component of \( z(p) \) is negative
    and at least one is positive, so the maximum of the components is
    strictly positive. In that case, every \( q\in f(p) \) has \( q^\ell=0 \) for
    some \( \ell \), so \( f(p) \subseteq \text{Boundary }\Delta \).
    In contrast, if \( z(p)=0 \), then \( z(p)\cdot q = 0 \) for all \( q \in \Delta \),
    so every \( q\in\Delta \) is a maximiser and \( f(p)=\Delta \).

    \paragraph{Step 2. Construction on Boundary \( \Delta \).}
    For \( p \in \text{Boundary }\Delta \), define
    \[
        f(p)
        :=
        \{ q \in \Delta : p\cdot q = 0 \}
        =
        \{ q \in \Delta : q^\ell = 0 \text{ whenever } p^\ell > 0 \}.
    \]
    Because \( p \) has at least one component equal to zero, this set is not
    empty: we can place all the weight of \( q \) on goods whose price is
    zero under \( p \).

    Note that with this construction no price vector on
    \( \text{Boundary }\Delta \) can be a fixed point of \( f(\cdot) \).
    Indeed, if \( p \in \text{Boundary }\Delta \) and \( p \in f(p) \), then
    we would have \( p\cdot p = 0 \), which is impossible because
    \( p \in \Delta \) implies \( p\cdot p>0 \).

    \paragraph{Step 3. Any fixed point of \( f \) is an equilibrium price.}
    Suppose that \( p^* \in \Delta \) is a fixed point of \( f \), i.e.
    \( p^* \in f(p^*) \).
    From Step~2 we know that no point on \( \text{Boundary }\Delta \) can be a
    fixed point, so we must have \( p^* \notin \text{Boundary }\Delta \),
    that is, \( p^* \in \text{Interior }\Delta \) and \( p^* > 0 \).

    If \( z(p^*) \neq 0 \), then Step~1 tells us that
    \( f(p^*) \subseteq \text{Boundary }\Delta \), so \( p^* \) cannot belong to
    \( f(p^*) \), a contradiction.
    Hence it must be that \( z(p^*) = 0 \).
    In other words, any fixed point of \( f(\cdot) \) is a price vector with
    zero aggregate excess demand.

    \paragraph{Step 4. The correspondence \( f \) is convex–valued and upper hemicontinuous.}
    \emph{Convex–valuedness.}
    When \( p \in \text{Interior }\Delta \), the set \( f(p) \) is the set of
    maximisers of a linear function \( q \mapsto z(p)\cdot q \) over the
    convex set \( \Delta \), hence it is convex.
    When \( p \in \text{Boundary }\Delta \), the set
    \( f(p) = \{q \in \Delta : p\cdot q=0\} \) is the intersection of the
    simplex \( \Delta \) with a linear subspace, and is therefore convex.
    Thus \( f(p) \) is convex for all \( p\in\Delta \).

    \smallskip
    \emph{Upper hemicontinuity.}
    Take any sequence \( p^n \to p \) in \( \Delta \) and a sequence
    \( q^n \in f(p^n) \) with \( q^n \to q \).
    We must show that \( q \in f(p) \).

    \medskip
    \noindent
    \textit{Case 1: \( p \in \text{Interior }\Delta \).}
    For \( n \) large enough, \( p^n \) is also in \( \text{Interior }\Delta \) and
    \( q^n \) maximises \( z(p^n)\cdot q \) over \( q\in\Delta \).
    The continuity of \( z(\cdot) \) gives \( z(p^n)\to z(p) \), and a standard
    argument for maximisers of continuous linear functionals on compact
    sets shows that any limit point of maximisers is a maximiser.
    Hence \( q \) maximises \( z(p)\cdot q \) over \( \Delta \), i.e.\ \( q \in f(p) \).

    \medskip
    \noindent
    \textit{Case 2: \( p \in \text{Boundary }\Delta \).}
    We need to show that \( p\cdot q = 0 \).
    Pick any good \( \ell \) with \( p^\ell>0 \).
    For \( n \) large enough we have \( p^{n,\ell} > 0 \) as well.

    First suppose that \( p^n \in \text{Boundary }\Delta \) for all large \( n \).
    Then by the definition of \( f(p^n) \) we have \( p^n\cdot q^n = 0 \) for
    such \( n \). Taking limits yields \( p\cdot q = 0 \).

    The more delicate case is when some \( p^n \) lie in
    \( \text{Interior }\Delta \) and converge to \( p \) on Boundary \( \Delta \).
    For those \( n \), \( q^n \) maximises \( z(p^n)\cdot q \) over \( \Delta \).
    We claim that, for large \( n \), \( q^{n,\ell}=0 \) whenever \( p^\ell>0 \).
    Once this is shown, passing to the limit gives \( q^\ell = 0 \) for all
    \( \ell \) with \( p^\ell>0 \), and hence \( p\cdot q = 0 \).

    Fix such an \( \ell \) with \( p^\ell>0 \).
    Because \( p^\ell>0 \), there is an \( \varepsilon>0 \) such that
    \( p^{n,\ell} \ge \varepsilon \) for all large \( n \).
    For those \( n \), optimality of \( q^n \) implies
    \[
        z^\ell(p^n)
        \le \max_k z^k(p^n),
    \]
    and by property~(v) of Proposition~\ref{prop:excess},
    the right–hand side tends to \( +\infty \) as \( n\to\infty \)
    whenever \( p^n \) approaches the boundary.
    On the other hand, the lower bound on excess demand in
    property~(iv) bounds \( z^\ell(p^n) \) from below, and
    Walras' law implies
    \[
        z^\ell(p^n)
        = - \frac{1}{p^{n,\ell}}
        \sum_{r\neq \ell} p^{n,r} z^r(p^n)
        \le \frac{s}{\varepsilon},
    \]
    where \( s>0 \) is the bound from property~(iv).
    Thus \( z^\ell(p^n) \) is bounded above, which is only compatible with
    property~(v) if \( q^{n,\ell}=0 \) for all large \( n \).
    (Intuitively, as \( p^n \) approaches the boundary, any maximal excess
    demand must be concentrated on goods whose prices go to zero.)

    We conclude that in the limit \( q \) assigns positive weight only to
    goods whose prices under \( p \) are zero, so \( p\cdot q = 0 \) and hence
    \( q \in f(p) \).

    In both cases \( q \in f(p) \), so \( f \) is upper hemicontinuous.

    \paragraph{Step 5. Existence of a fixed point.}
    The simplex \( \Delta \) is nonempty, compact, and convex.
    By the previous step, the correspondence
    \( f : \Delta \rightrightarrows \Delta \) has nonempty, convex values
    and is upper hemicontinuous.
    Theorem \ref{thm:kaku} therefore guarantees the existence of
    a fixed point \( p^* \in \Delta \) with \( p^* \in f(p^*) \).

    By Step~3, any such fixed point satisfies \( z(p^*) = 0 \).
    Hence there exists a price vector \( p^* \in \mathbb{R}^\ell_{++} \) such
    that \( z(p^*) = 0 \).
    By Proposition~\ref{prop:walr_excess}, this means that a Walrasian
    equilibrium exists.
\end{proof}

\paragraph{Things to read.} This lecture is based on \citet[pp. 578-586]{mas-colellMicroeconomicTheory1995}.

\section{Exercises}

\begin{exercise}
    Show that if demand equals supply in \( k-1 \) markets, then it also equals supply in the \( k \)-th market. (Hint: Use Walras' law.)
\end{exercise}

\begin{exercise}
    Prove Proposition \ref{prop:walr_excess}.
\end{exercise}

\begin{exercise}
    Prove property 1. of the excess demand function in Proposition \ref{prop:excess}.
\end{exercise}

\bibliographystyle{apacite}  % or another  style
\bibliography{references} % .bib file goes in ./bib/