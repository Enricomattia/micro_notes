\renewcommand{\thefootnote}{\fnsymbol{footnote}}

\chapter[Existence of competitive equilibria]%
 {Existence of competitive equilibria}%
\label{ch:L10}

% 3) Reset things so later footnotes go back to 1, 2, 3, …
%\setcounter{footnote}{0}
\renewcommand{\thefootnote}{\arabic{footnote}}

We have defined Walrasian (competitive) equilibrium and discussed its properties. In this lecture, we will prove that a competitive equilibrium exists under certain assumptions on preferences. We assume preferences are continuous, strictly convex, and strongly monotone. These are strengthening of assumptions we have made before.

\begin{definition}
    A preference relation is \textbf{strongly monotone} if for all \(x, y \in \mathbb{R}^L_+\) such that \(x \geq y\) and \(x \neq y\), we have \(x \succ y\).
\end{definition}

\begin{definition}
    A preference relation is \textbf{strictly convex} if for all \(x, y, z \in \mathbb{R}^L_+\) such that \(y \succ x\) and \(z \succ x\), we have for all \(\alpha \in (0,1)\), \(\alpha y + (1-\alpha) z \succ x\). Equivalently, all upper contour sets are strictly convex sets.
\end{definition}

If a preference relation is continuous, strictly convex, and strongly monotone, then the Walrasian demand is single-valued and can therefore be viewed as a function. We can then define the excess demand function as follows.

\begin{definition}
    The \textbf{excess demand function} of and individual \( i \) with a single-valued Walrasian demand function \( D_i(p, e_i) \) is given by
    \[z_i(p, e_i) = D_i(p, e_i) - e_i.\]
\end{definition}

From the individual excess demand functions, we can construct the aggregate excess demand function.

\[
    z(p) = \sum_i z_i (p) .
\]

The excess demand function maps prices to allocations. Under our assumptions on preferences, Walrasian equilibrium can be characterised through the excess demand function as follows.

\begin{proposition}\label{prop:walr_excess}
    If individual preferences \( \succsim_i \) are continuous, strictly convex, and strongly monotone for each \( i \), then an allocation \( x \) is a Walrasian equilibrium if and only if there exists a price vector \( p \) such that \( z (p) = 0 \).
\end{proposition}

The excess demand function has several important properties that we will use to prove the existence of a competitive equilibrium.

\begin{proposition}\label{prop:excess}
    If individual preferences \( \succsim_i \) are continuous, strictly convex, and strongly monotone for each \( i \), then the aggregate excess demand function \( z(p) \) satisfies the following properties:
    \begin{enumerate}
        \item \( z(p) \) is \textbf{homogeneous of degree zero}:\( z( \alpha p) = z(p) \) for all \( \alpha > 0 \);
        \item \( z(p) \) satisfies Walras' law: \( p \cdot z(p) = 0 \) for all strictly positive prices \( p \);
        \item \( z(p) \) is continuous;
        \item there is an \( s > 0 \) such that for all goods \( \ell \) and prices \( p \), \( z_{\ell} (p) > -s \);
        \item if \(p^n \) is a sequence of prices converging to \( p \) with \( p_{\ell} = 0 \) for some good \( \ell \), then \( \max \{ z^1 (p^n), \ldots, z^L (p^n) \} \to +\infty \).
    \end{enumerate}
\end{proposition}

The existence result we study here relies on Kakutani's fixed–point theorem.

\begin{theorem}\label{thm:kaku} (\textbf{Kakutani's fixed point})
    Let \( A \subseteq \mathbb{R}^L \) be a non-empty, compact, and convex set, and \( f : A \rightrightarrows A \) be an upper hemicontinuous correspondence such that \( f(x) \subseteq A \) is non-empty and convex for each \( x \in A \). Then, \( f \) has a fixed point, i.e., there exists \( x \in A \) such that \( x \in f(x) \).
\end{theorem}

The existence result is the following.

\begin{proposition}
    If the aggregate excess demand function \( z(p) \) satisfies the properties in Proposition \ref{prop:excess}, then there exists a price vector \( p \) such that \( z(p) = 0 \). Therefore, in such economy a Walrasian equilibrium exists.
\end{proposition}

\begin{proof}
    Check \citet[p. 586]{mas-colellMicroeconomicTheory1995}.
\end{proof}

\paragraph{Things to read.} This lecture is based on \citet[pp. 578-587]{mas-colellMicroeconomicTheory1995}.

\section{Exercises}

\begin{exercise}
    Show that if demand equals supply in \( k-1 \) markets, then it also equals supply in the \( k \)-th market. (Hint: Use Walras' law.)
\end{exercise}

\begin{exercise}
    Prove Proposition \ref{prop:walr_excess}.
\end{exercise}

\begin{exercise}
    Prove property 1. of the excess demand function in Proposition \ref{prop:excess}.
\end{exercise}

\bibliographystyle{apacite}  % or another  style
\bibliography{references} % .bib file goes in ./bib/