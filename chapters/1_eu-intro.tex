\renewcommand{\thefootnote}{\fnsymbol{footnote}}

\chapter[Introduction to Expected Utility]%
 {Introduction to Expected Utility}
\label{ch:L1}

% 3) Reset things so later footnotes go back to 1, 2, 3, …
\setcounter{footnote}{0}
\renewcommand{\thefootnote}{\arabic{footnote}}

\section{How to model uncertainty}\label{sec:L1-intro}

Let's start by thinking about how we represent uncertainty. Say that you did a bet with a friend of yours, if a fair coin toss results in head, you get \(10\) euros, otherwise you pay \(10\) euros to your friend. There are two outcomes, \(10\) and \(-10\), and since the coin is fair, each realises with a probability of \(\frac{1}{2}\). The describe this simple example, we started from a set of outcomes, monetary transfers, and detailed the probability of each outcome realising. I refer to such object as a \textit{lottery}. Denote the set of outcomes with \(X\), generic elements of \(X\) are \(x,y,z\). For now, we assume that \(X\) contains only a finite set of elements. Outcomes are not enough to describe a lottery, we need to consider probability distributions over outcomes, as the \(\frac{1}{2}\)-\(\frac{1}{2}\) distribution of the fair coin above. The set of lotteries over the set of outcomes \(X\) is denoted with \(\Delta (X)\). Each element of \(\Delta(X)\) is a function \(p : X \rightarrow [0,1]\) such that \(\sum_{x \in X} p(x) =1\) mapping each outcome \(x\) to a number \(p(x)\) between \(0\) and \(1\), representing the probability \(x\) realises, so that the sum of all these probabilities is \(1\).\footnote{Why do we write the sum \(\sum_{x \in X} p(x) =1\) instead of an integral?}

\begin{example}
	In the example above, the set of outcomes is \(\{10,-10\}\) and the lottery induced by the fair coin toss is such that \(p(10) =  p(-10) =\frac{1}{2}\).
\end{example}

Check \cite{krepsNotesTheoryChoice1988}.

\bibliographystyle{apacite}  % or another  style
\bibliography{references} % .bib file goes in ./bib/