\renewcommand{\thefootnote}{\fnsymbol{footnote}}

\chapter[General equilibrium theory]%
 {General equilibrium theory}%
\label{ch:L7}

% 3) Reset things so later footnotes go back to 1, 2, 3, …
%\setcounter{footnote}{0}
\renewcommand{\thefootnote}{\arabic{footnote}}

\section{Exchange economies}\label{sec:L7-intro}

\paragraph{Primitives and definitions.} We now generalise the example in Section~\ref{sec:L6-example}. There is a set of \( n \) individuals \( I = \{1,\dots,n\} \) and the consumption space \( \mathbb{R}^{\ell}_{+} \). Each individual has an endowment \( e_i \in \mathbb{R}^{\ell}_{+} \), where \( e_i = (e_i^{1}, \dots, e_i^{\ell}) \). The total endowment in the economy is

\[
	\sum_{i} e_i = \bar e .
\]

Each individual has preferences \( \succsim_i \) over \( \mathbb{R}^{\ell}_{+} \). A generic consumption bundle for individual \( i \) is \( x_i = (x_i^{1}, \dots, x_i^{\ell}) \). We always assume that preferences are complete and transitive for each \( i \).

An \textbf{economy} is a profile \( E = (\succsim_i, e_i)_{i \in I} \). An allocation \( x \) is \textbf{feasible} for \( E \) if

\[
	\sum_{i} x_i \le \bar e .
\]

Feasible allocations are elements of \( \mathbb{R}^{\ell n}_{+} \). A price vector is \( p = (p^{1}, \dots, p^{\ell}) \in \mathbb{R}^{\ell}_{+} \), which assigns a price to each good. For each price vector, we define the budget set of an individual with endowment \( e_i \).

\begin{definition}
	Given endowment \( e_i \) and prices \( p \), the budget set is

	\[
		B(p, e_i)
		= \left\{
		x_i \in \mathbb{R}^{\ell}_+ \;\middle|\; p \cdot x_i \le p \cdot e_i
		\right\}.
	\]
\end{definition}

Note that the budget set is always convex and, when all prices are strictly positive, it is also compact. We will often assume prices are strictly positive. Given prices \( p \), endowment \( e_i \), and preference relation \( \succsim_i \), we define the Walrasian demand.

\begin{definition}
	The \textbf{Walrasian demand} of \( i \), given endowment \( e_i \) and prices \( p \), is
	\[
		D_i(p, e_i)
		= \left\{
		x_i \in B(p, e_i) \;\middle|\;
		x_i \succsim_i x'_i \ \text{for all} \ x'_i \in B(p, e_i)
		\right\}.
	\]
\end{definition}

The Walrasian demand is the set of most preferred bundles in the budget set.

Given a preference relation \( \succsim_i \), we define the \textbf{upper and lower contour sets} at bundle \( x_i \):

\[
	U_i(x_i)
	:= \{\, x'_i \in \mathbb{R}^{\ell}_{+} \mid x'_i \succsim_i x_i \,\}
	\quad \text{and} \quad
	L_i(x_i)
	:= \{\, x'_i \in \mathbb{R}^{\ell}_{+} \mid x_i \succsim_i x'_i \,\} .
\]

A bundle \( x_i' \) is in the upper contour set of \( x_i \) if it is weakly preferred to \( x_i \); it is in the lower contour set if it is weakly dispreferred to \( x_i \). These sets are useful for defining some properties of preference relations. They are illustrated in Figure~\ref{fig:contour-sets}, assuming that \( \succsim_i \) is increasing.

\begin{figure}[H]
	\begin{center}
		\begin{tikzpicture}[x=0.70pt,y=0.70pt,yscale=-1,xscale=1]
			%uncomment if require: \path (0,469); %set diagram left start at 0, and has height of 469

			%Shape: Axis 2D [id:dp6363214867054915] 
			\draw  (143,368.12) -- (491,368.12)(177.8,74) -- (177.8,400.8) (484,363.12) -- (491,368.12) -- (484,373.12) (172.8,81) -- (177.8,74) -- (182.8,81)  ;
			%Curve Lines [id:da3683132673609393] 
			\draw    (213,100.6) .. controls (205,256.6) and (299,352.6) .. (454,339.6) ;
			%Shape: Circle [id:dp9112121344081351] 
			\draw  [fill={rgb, 255:red, 0; green, 0; blue, 0 }  ,fill opacity=1 ] (273.4,286.6) .. controls (273.4,284.94) and (274.74,283.6) .. (276.4,283.6) .. controls (278.06,283.6) and (279.4,284.94) .. (279.4,286.6) .. controls (279.4,288.26) and (278.06,289.6) .. (276.4,289.6) .. controls (274.74,289.6) and (273.4,288.26) .. (273.4,286.6) -- cycle ;

			% Text Node
			\draw (481,380.4) node [anchor=north west][inner sep=0.75pt]    {$x^{1}$};
			% Text Node
			\draw (138,68.4) node [anchor=north west][inner sep=0.75pt]    {$x^{2}$};
			% Text Node
			\draw (464,329.4) node [anchor=north west][inner sep=0.75pt]    {$\succsim _{i}$};
			% Text Node
			\draw (284,263.4) node [anchor=north west][inner sep=0.75pt]    {$x_{i}$};
			% Text Node
			\draw (312,209.4) node [anchor=north west][inner sep=0.75pt]    {$U_i( x_{i})$};
			% Text Node
			\draw (207,304.4) node [anchor=north west][inner sep=0.75pt]    {$L_i( x_{i})$};
		\end{tikzpicture}
		\caption{Upper and lower contour sets at bundle \( x_i \).}
		\label{fig:contour-sets}
	\end{center}
\end{figure}

\paragraph{Properties of preferences.} As we did with preferences over lotteries, we now define some properties of preferences over consumption bundles. Some of the results we present later require these properties. Note that we are studying preferences over \( \mathbb{R}^{\ell}_{+} \), which has a structure we can exploit. However, we no longer have lotteries, so we cannot use properties that rely on that structure, for example independence.\footnote{However, in general equilibrium under uncertainty the lottery structure is important, see \citet[ch. 19]{mas-colellMicroeconomicTheory1995}.}

\begin{definition}\label{ax:lns}
	\labelname{axn:lns}{Local non-satiation}
	A preference relation \( \succsim_i \) is \textbf{locally non-satiated} if, for every \( x_i \in \mathbb{R}^{\ell}_{+} \) and every \( \varepsilon > 0 \), there exists \( x_i' \in \mathbb{R}^{\ell}_{+} \) such that \( \|x_i' - x_i\| < \varepsilon \) and \( x_i' \succ_i x_i \).
\end{definition}

\usename{axn:lns} says that in every neighbourhood of every bundle there is another bundle that is strictly preferred. It is a weak form of monotonicity. In fact, monotonicity implies \usename{axn:lns}, but the converse is not true. It rules out the case in which all goods are \textit{bads}, in the sense that individuals do not like them. \usename{axn:lns} also rules out thick indifference curves. Consider the preferences represented in Figure~\ref{fig:satiation}. These preferences are not locally non-satiated at bundle \( x_i \), since in any neighbourhood of \( x_i \) there is no strictly preferred bundle.

\begin{figure}[H]
	\begin{center}
		\begin{tikzpicture}[x=0.75pt,y=0.75pt,yscale=-1,xscale=1]
			%uncomment if require: \path (0,502); %set diagram left start at 0, and has height of 502

			%Shape: Axis 2D [id:dp20007957527937936] 
			\draw  (143,375.12) -- (491,375.12)(177.8,81) -- (177.8,407.8) (484,370.12) -- (491,375.12) -- (484,380.12) (172.8,88) -- (177.8,81) -- (182.8,88)  ;
			%Curve Lines [id:da3724834131003195] 
			\draw    (225,95.6) .. controls (217,251.6) and (311,347.6) .. (466,334.6) ;
			%Curve Lines [id:da5809305123487184] 
			\draw    (256,64.6) .. controls (248,220.6) and (342,316.6) .. (497,303.6) ;
			%Straight Lines [id:da4600562742754496] 
			\draw    (225,95.6) -- (256,64.6) ;
			%Straight Lines [id:da003738031499627703] 
			\draw    (466,334.6) -- (497,303.6) ;
			%Shape: Circle [id:dp6153634380448432] 
			\draw  [fill={rgb, 255:red, 0; green, 0; blue, 0 }  ,fill opacity=1 ] (286.4,255.6) .. controls (286.4,253.94) and (287.74,252.6) .. (289.4,252.6) .. controls (291.06,252.6) and (292.4,253.94) .. (292.4,255.6) .. controls (292.4,257.26) and (291.06,258.6) .. (289.4,258.6) .. controls (287.74,258.6) and (286.4,257.26) .. (286.4,255.6) -- cycle ;
			%Shape: Circle [id:dp7656031105045924] 
			\draw   (275.4,255.6) .. controls (275.4,247.87) and (281.67,241.6) .. (289.4,241.6) .. controls (297.13,241.6) and (303.4,247.87) .. (303.4,255.6) .. controls (303.4,263.33) and (297.13,269.6) .. (289.4,269.6) .. controls (281.67,269.6) and (275.4,263.33) .. (275.4,255.6) -- cycle ;

			% Text Node
			\draw (481,387.4) node [anchor=north west][inner sep=0.75pt]    {$x^{1}$};
			% Text Node
			\draw (138,75.4) node [anchor=north west][inner sep=0.75pt]    {$x^{2}$};
			% Text Node
			\draw (480,332.4) node [anchor=north west][inner sep=0.75pt]    {$\succsim _{i}$};
			% Text Node
			\draw (305.4,259) node [anchor=north west][inner sep=0.75pt]    {$x_{i}$};
		\end{tikzpicture}
		\caption{Preferences that are not locally non-satiated at bundle \( x_i \).}
		\label{fig:satiation}
	\end{center}
\end{figure}

A second important property is convexity.

\begin{definition}\label{def:convexity}
	\labelname{axn:convexity}{Convexity}
	A preference relation \( \succsim_i \) is \textbf{convex} if for all
	\( x_i, x'_i, x''_i \in \mathbb{R}^{\ell}_{+} \),
	whenever \( x'_i \succsim_i x_i \) and \( x''_i \succsim_i x_i \), then
	\[
		\alpha x'_i + (1 - \alpha) x''_i \succsim_i x_i
		\quad \text{for all } \alpha \in [0,1].
	\]
\end{definition}

\usename{axn:convexity} is equivalent to upper contour sets being convex.\footnote{You are asked to show this in Exercise~\ref{ex:convexity}.} It captures a form of \textit{diminishing marginal returns}. It can also be viewed as a preference for \textit{diversification}.

Finally, we define continuity.

\begin{definition}\label{def:cont-pref}
	\labelname{axn:continuity1}{Continuity}
	A preference relation \( \succsim_i \) on \( \mathbb{R}^{\ell}_{+} \) is \textbf{continuous} if for every bundle \( x_i \in \mathbb{R}^{\ell}_{+} \), both the upper and lower contour sets of \( x_i \) are closed.
\end{definition}

\usename{axn:continuity1} says that small changes in consumption bundles do not lead to jumps in preferences. If you are wondering about the relationship between this notion of continuity and the notion of \usename{axn:continuity} over lotteries from Chapter~\ref{ch:L2}, the answer is that the latter is \textit{weaker} than the former. However, under the independence axiom for preferences over lotteries, they are equivalent.\footnote{Actually, the notion of Archimedean continuity in Lecture~\ref{ch:L2} is equivalent to continuity in Definition~\ref{def:cont-pref} even under a specific weakening of independence \citep{karniArchimedeanContinuity2007}.}

If a preference relation \( \succsim_i \) is continuous, then for every endowment \( e_i \) and every strictly positive price vector \( p \in \mathbb{R}^{\ell}_{++} \), the Walrasian demand \( D_i(p, e_i) \) is non-empty. This is because a complete, transitive, and continuous preference relation admits a continuous utility representation \citep[Proposition 3.C.1]{mas-colellMicroeconomicTheory1995}.\footnote{Interestingly, the result dates back to \cite{cantorContributionsFoundingTheory1915}; see the discussion in \citet[ch. 6.1]{gilboaTheoryDecisionUncertainty2009}.} The budget set \( B(p, e_i) \) is compact when prices are strictly positive, so the continuous utility function attains a maximum on it. If preferences are also locally non-satiated, then the Walrasian demand \( D_i(p, e_i) \) lies on the budget line \( \{ x_i \in \mathbb{R}^{\ell}_{+} \mid p \cdot x_i = p \cdot e_i \} \).\footnote{You are asked to show this in Exercise~\ref{ex:lns}.}

\begin{techremark}
	You might wonder whether there exists a reasonable-looking preference relation that is not continuous. The answer is yes. Consider the \textbf{lexicographic} preference relation on \( \mathbb{R}^{2}_{+} \): for any two bundles \( x_i = (x_i^{1}, x_i^{2}) \) and \( x_i' = (x_i'^{1}, x_i'^{2}) \), define \( x_i' \succsim_i x_i \) if either (i) \( x_i'^{1} > x_i^{1} \), or (ii) \( x_i'^{1} = x_i^{1} \) and \( x_i'^{2} \ge x_i^{2} \). This preference relation is complete and transitive, but it is not continuous. If you want, you can have fun showing this (or check \citet[p. 47]{mas-colellMicroeconomicTheory1995}).
\end{techremark}

\section{Allocations, rules and their properties}

We have introduced the primitives of an exchange economy of interest. Given these primitives, we ask which allocations we can select in this economy. Such an allocation should have two types of properties: first, it should satisfy some normative criteria, for example efficiency or distributional fairness; second, it should be compatible with individuals' incentives, otherwise we could simply force an allocation while disregarding individual preferences.

The efficiency criterion we use is \textbf{Pareto optimality}.

\begin{definition}\label{ax:po}
	\labelname{axn:po}{Pareto optimal}
	A feasible allocation \( x \) is \textbf{Pareto optimal} if there is no feasible allocation \( x' \) with \( x'_i \succsim_i x_i \) for all \( i \) and \( x'_j \succ_j x_j \) for some \( j \).
\end{definition}

Pareto optimality says that a feasible allocation is Pareto optimal if there is no other feasible allocation that makes everyone weakly better off and at least one individual strictly better off.

The distributional criterion we consider is \textbf{no-envy}.

\begin{definition}\label{ax:no-envy}
	\labelname{axn:no-envy}{No-envy}
	An allocation \( x \) satisfies \textbf{no-envy} if for all individuals \( i \) and \( j \),

	\[
		x_i \succsim_i x_j.
	\]
\end{definition}

An allocation satisfies \usename{axn:no-envy} if no individual prefers the bundle of another individual to his own bundle.

For individual incentives, we consider \textbf{Walrasian equilibrium}.

\begin{definition}\label{ax:weq}
	\labelname{axn:weq}{Walrasian equilibrium}
	A feasible allocation \( x \) is a \textbf{Walrasian equilibrium} if there exist strictly positive prices \( p \in \mathbb{R}^{\ell}_{++} \) such that, for all individuals \( i \),

	\[
		x_i \in D_i(p, e_i).
	\]
\end{definition}

An allocation is a \usename{axn:weq} if there exist prices such that each individual would consume a bundle in his Walrasian demand given his endowment and those prices. Notice that we are not saying that these prices exist and that individuals actually face them. Instead, we are detailing a property of an allocation: can such an allocation be, in principle, the result of an individual optimisation problem in the style of choice theory? If yes, then the allocation is a \usename{axn:weq}.

We will consider a couple of variants of \usename{axn:weq}. A \textbf{Walrasian equilibrium with transfers} is a generalisation in which we allow transfers among individuals before they optimise.

\begin{definition}\label{ax:weqt}
	\labelname{axn:weqt}{Walrasian equilibrium with transfers}
	A feasible allocation \( x \) is a \textbf{Walrasian equilibrium with transfers} if there exist strictly positive prices \( p \in \mathbb{R}^{\ell}_{++} \) and transfers \( (T_i)_{i\in I} \) satisfying

	\[
		\sum_{i} T_i = 0
		\quad\text{and}\quad
		e_i + T_i \in \mathbb{R}^{\ell}_{+}\ \text{for all } i,
	\]
	such that
	\[
		x_i \in D_i(p, e_i + T_i) \text{ for all } i .
	\]
\end{definition}

An allocation is a \usename{axn:weqt} if there exist prices and transfers such that each individual would consume a bundle in his Walrasian demand given his adjusted endowment and those prices. Each \usename{axn:weq} is also a \usename{axn:weqt} with no transfers, that is, \( T_i = 0 \) for all \( i \).

The last equilibrium notion we consider is \textbf{Egalitarian Walrasian equilibrium}.

\begin{definition}\label{ax:weqe}
	\labelname{axn:weqe}{Egalitarian Walrasian equilibrium}
	A feasible allocation \( x \) is an \textbf{Egalitarian Walrasian equilibrium} if there exist strictly positive prices \( p \in \mathbb{R}^{\ell}_{++} \) such that

	\[
		x_i \in D_i \left(p,\frac{\bar e}{n}\right) \text{ for all } i .
	\]
\end{definition}

An allocation is a \usename{axn:weqe} if there exist prices such that each individual would consume a bundle in his Walrasian demand given an equal share of the total endowment and those prices. Suppose we collect all the resources in the economy and redistribute them equally among individuals before they optimise. If the resulting allocation is a \usename{axn:weq}, it is a \usename{axn:weqe}.

Let \( F(E) \) be the set of feasible allocations. Which feasible allocations do we want to select? To answer this question, we introduce allocation rules. An \textbf{allocation rule} \( R \) maps an economy \( E \) to a subset of feasible allocations \( R(E) \subseteq F(E) \). The standard axiomatic approach is to introduce assumptions on \( R \) and see what allocations it induces. For example, we might impose that \( R \) only selects Pareto optimal allocations. In what follows, we study allocation rules that select allocations with the efficiency, distributional, and incentive properties we introduced above.

\begin{definition}
	An allocation rule \( R^{W} \) is \textbf{Walrasian} if for all economies \( E \) it selects allocations that are Walrasian equilibria; that is, if for all \( E \)
	\[
		R^{W}(E)
		=
		\left\{
		x \in F(E) \;\middle|\; \exists p \in \mathbb{R}_{++}^{\ell} \text { s.t. } x_i \in D_i(p, e_i) \text{ for all } i
		\right\}.
	\]
\end{definition}

An allocation rule is Walrasian if it selects all Walrasian equilibria of the economy. Equivalently, we can define allocation rules that select all Walrasian equilibria with transfers and all Egalitarian Walrasian equilibria.

\begin{definition}
	An allocation rule \( R^{WT} \) is \textbf{Walrasian with transfers} if for all economies \( E \) it selects allocations that are Walrasian equilibria with transfers; that is, if for all \( E \)

	\[
		R^{WT}(E)
		=
		\left\{
		x \in F(E) \;\middle|\;
		\exists\, p \in \mathbb{R}^{\ell}_{++},\ \exists\, (T_i)_{i}
		\ \text{s.t.}
		\ \begin{array}{l}
			\\[2pt]
			e_i + T_i \in \mathbb{R}^{\ell}_{+} \ \text{for all } i, \ \sum_i T_i = 0 \\[2pt]
			x_i \in D_i(p, e_i + T_i) \ \text{for all } i
		\end{array}
		\right\}.
	\]
\end{definition}

\begin{definition}
	An allocation rule \( R^{EW} \) is \textbf{Egalitarian Walrasian} if for all economies \( E \) it selects allocations that are Egalitarian Walrasian equilibria; that is, if for all \( E \)

	\[
		R^{EW}(E)
		=
		\left\{
		x \in F(E) \;\middle|\; \exists p \in \mathbb{R}_{++}^{\ell} \text { s.t. } x_i \in D_i\!\left(p,\frac{\bar e}{n}\right) \text{ for all } i
		\right\}.
	\]
\end{definition}

In the next lectures, we will study properties of these allocation rules and the relationships between them.

\begin{techremark}
	A question you might ask to better understand these definitions is: what exactly are allocations? It is easier to think first about a static problem, where allocations are just consumption bundles. However, in principle, an allocation might encode time and uncertainty. A consumption bundle might be a stream of consumption over time, or a contingent consumption plan over states of the world. We are simply not making this structure explicit. A deeper discussion of this point is in \citet[ch. 2]{debreuTheoryValueAxiomatic1959}.
\end{techremark}

\paragraph{Things to read.} Properties of preferences are discussed in detail in \citet[ch. 3]{mas-colellMicroeconomicTheory1995}. Exchange economies are introduced in \citet[ch. 2]{hildenbrandIntroductionEquilibriumAnalysis1976}. You can also read \citet[ch. 16]{mas-colellMicroeconomicTheory1995} if you wish, but it moves straight to production economies. A treatment close to the approach in these notes is in \cite{thomsonFairAllocationRules2011}. A brief discussion of various Egalitarian Walrasian allocation rules is in \citet[ch. 1]{fleurbaeyTheoryFairnessSocial2011}. \cite{debreuTheoryValueAxiomatic1959} is a classic reference with historical value. You can find everything we discuss there.

\section{Exercises}

\begin{exercise}\label{ex:convexity}
	Show that convexity in Definition~\ref{def:convexity} is equivalent to the following statement: for every bundle \( x_i \in \mathbb{R}^{\ell}_{+} \), both the upper and lower contour sets of \( x_i \) are convex.
\end{exercise}

\begin{exercise}\label{ex:lns}
	Show that if a preference relation \( \succsim_i \) is locally non-satiated, then for every endowment \( e_i \) and every strictly positive price vector \( p \in \mathbb{R}^{\ell}_{++} \), the Walrasian demand \( D_i(p, e_i) \) lies on the budget line \( \{ x_i \in \mathbb{R}^{\ell}_{+} \mid p \cdot x_i = p \cdot e_i \} \).
\end{exercise}

\bibliographystyle{apacite}  % or another  style
\bibliography{references} % .bib file goes in ./bib/