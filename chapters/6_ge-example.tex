\renewcommand{\thefootnote}{\fnsymbol{footnote}}

\chapter[Exchange economies]%
 {Exchange economies}%
\label{ch:L6}

% 3) Reset things so later footnotes go back to 1, 2, 3, …
%\setcounter{footnote}{0}
\renewcommand{\thefootnote}{\arabic{footnote}}

\section{An illustrative example}\label{sec:L6-intro}

Ann and Bob, two individuals, consume two goods, apples and bananas.

First, we check each individual consumption space, assume monotonic preferences, and draw strictly convex indifference curves.

Then, we introduce prices and budgets, we see the budget set together with the indifference curves, and we discuss the optimum consumption.

We briefly discuss properties of indifference curves.

We introduce endowments for each individual.

Then, we put everything in a single graph, the Edgeworth box. We notice that the individual budget set depends on his endowment and on prices.

We show an instance of budget and demand in which the two preferred allocations are not compatible.

Then, we show a walrasian equilibrium, in which demands are compatible.

We show that when prices are doubled, also wealth is doubled, and the budget set and demand remain the same.

Describe a strange situation in which there are no walrasian equilibria.

Finally, we introduce the concept of Pareto optimality, we show in the Edgeworth box allocations that are and are not Pareto optimal.

Discuss the Pareto set and the contract curve.

Introduce first and second welfare theorems.

Discuss no-envy.

\paragraph{Things to read.} It might be useful for you to review (or study, if you never encountered these topics before), \citet[pp. 51-70, 76-84]{hildenbrandIntroductionEquilibriumAnalysis1976}. If you want (and you \textquote{should want}) to go deeper, study study \citet[pp. 17–23, 40–56]{mas-colellMicroeconomicTheory1995}. \cite{arrowSocialChoiceIndividual2012}.

\section{Exercises}

hey.

\bibliographystyle{apacite}  % or another  style
\bibliography{references} % .bib file goes in ./bib/