\renewcommand{\thefootnote}{\fnsymbol{footnote}}

\chapter[Second theorem of welfare economics]%
 {Second theorem of welfare economics}%
\label{ch:L9}

% 3) Reset things so later footnotes go back to 1, 2, 3, …
%\setcounter{footnote}{0}
\renewcommand{\thefootnote}{\arabic{footnote}}

We now turn to the second theorem of welfare economics, which establishes that, under certain conditions, any Pareto optimal allocation can be implemented as a \usename{axn:weqt}. The proof relies on convexity of preferences and uses a geometric argument based on supporting hyperplanes. Supporting and separating hyperplane arguments are common in economic theory.\footnote{As an example, a proof of the expected utility representation in Lecture \ref{ch:L2} using a separation argument is in \citet[ch.~8.3.3]{gilboaTheoryDecisionUncertainty2009}.} In particular, we will invoke the specific result stated below.\footnote{If you are interested, a nice source is \citet[ch.~11]{rockafellarConvexAnalysis1970}.} Recall that the interior of a set \(X \subset \mathbb{R}^{\ell}\), denoted \(\operatorname{int} X\), is the set of all points \(x \in X\) for which there exists an open ball around \(x\) that is fully contained in \(X\).

\begin{theorem}\label{thm:supporting-hyperplane} (\textbf{Supporting hyperplane})
    Let \(X \subset \mathbb{R}^{\ell}\) be a convex set and let \(x \notin \operatorname{int} X\). Then there exists \(p \in \mathbb{R}^{\ell}\) with \(p \neq 0\) such that
    \[
        p \cdot x \;\ge\; p \cdot y \qquad \text{for all } y \in X .
    \]
\end{theorem}

In words, there is a hyperplane with normal vector \(p\) passing through \(x\) such that \(X\) lies entirely in the weak half-space on one side of it. Figure \ref{fig:supporting-hyperplane} illustrates the theorem in two dimensions.

\begin{figure}[H]
    \begin{center}
        \begin{tikzpicture}[x=0.75pt,y=0.75pt,yscale=-1,xscale=1]
            %uncomment if require: \path (0,414); %set diagram left start at 0, and has height of 414

            %Shape: Axis 2D [id:dp9653789202234155] 
            \draw  (147,306.86) -- (473,306.86)(179.6,41) -- (179.6,336.4) (466,301.86) -- (473,306.86) -- (466,311.86) (174.6,48) -- (179.6,41) -- (184.6,48)  ;
            %Curve Lines [id:da15506102017650236] 
            \draw    (157,61.2) .. controls (364,127.2) and (378,189.2) .. (372,365.2) ;
            %Shape: Circle [id:dp463732573499136] 
            \draw  [fill={rgb, 255:red, 0; green, 0; blue, 0 }  ,fill opacity=1 ] (301.4,224.6) .. controls (301.4,222.94) and (302.74,221.6) .. (304.4,221.6) .. controls (306.06,221.6) and (307.4,222.94) .. (307.4,224.6) .. controls (307.4,226.26) and (306.06,227.6) .. (304.4,227.6) .. controls (302.74,227.6) and (301.4,226.26) .. (301.4,224.6) -- cycle ;
            %Shape: Circle [id:dp24331242668872677] 
            \draw  [fill={rgb, 255:red, 0; green, 0; blue, 0 }  ,fill opacity=1 ] (340.4,170.6) .. controls (340.4,168.94) and (341.74,167.6) .. (343.4,167.6) .. controls (345.06,167.6) and (346.4,168.94) .. (346.4,170.6) .. controls (346.4,172.26) and (345.06,173.6) .. (343.4,173.6) .. controls (341.74,173.6) and (340.4,172.26) .. (340.4,170.6) -- cycle ;
            %Straight Lines [id:da20808313465799966] 
            \draw    (289.67,83.99) -- (392.67,248.99) ;

            % Text Node
            \draw (238,190.4) node [anchor=north west][inner sep=0.75pt]    {$X$};
            % Text Node
            \draw (289,224.4) node [anchor=north west][inner sep=0.75pt]    {$y$};
            % Text Node
            \draw (316,155.4) node [anchor=north west][inner sep=0.75pt]    {$x$};
            % Text Node
            \draw (393,222.4) node [anchor=north west][inner sep=0.75pt]    {$p$};
            % Text Node
            \draw (460,317.4) node [anchor=north west][inner sep=0.75pt]    {$x^{1}$};
            % Text Node
            \draw (155,69.4) node [anchor=north west][inner sep=0.75pt]    {$x^{2}$};


        \end{tikzpicture}
        \caption{A supporting hyperplane through \(x\) for the convex set \(X\).}
        \label{fig:supporting-hyperplane}
    \end{center}
\end{figure}

The last ingredient we need is a strengthening of \usename{axn:local-nonsat}.

\begin{definition}
    A preference relation \(\succsim_i\) on \(\mathbb{R}^\ell_{+}\) is \textbf{monotonic} if for every \(x_i, x'_i \in \mathbb{R}^\ell_{+}\) with \(x'_i \ge x_i\) and \(x'_i \neq x_i\), we have \(x'_i \succ_i x_i\).
\end{definition}

A monotonic preference relation is one where (holding everything else fixed) more of any good is always strictly preferred to less. Monotonicity implies local non-satiation, but the converse is not true.

We can now state and prove the second fundamental theorem of welfare economics.

\begin{theorem}\label{thm:sftwe} (\textbf{Second welfare theorem})
    If all preferences in the economy \(E\) are monotonic, convex, and continuous, and each individual has a strictly positive endowment \(e_i \in \mathbb{R}^\ell_{++}\), then every interior Pareto optimal allocation \(x \in \mathbb{R}^{\ell n}_{++}\) can be supported as a Walrasian equilibrium with transfers. That is,
    \[
        x \text{ is Pareto optimal and } x \in \mathbb{R}^{\ell n}_{++}
        \quad \Longrightarrow \quad
        x \in R^{WT}(E).
    \]
\end{theorem}

\begin{proof}
    Say we want to implement the interior Pareto optimal allocation \( x \). Define transfers \( T_i \) by

    \[
        T_i = x_i - e_i \qquad\text{for each } i.
    \]

    Feasibility of \( x \) implies \( \sum_{i} x_i = \sum_{i} e_i \), hence \( \sum_{i} T_i = 0 \), so \( (T_i)_{i} \) is a feasible vector of lump–sum transfers. We have to find strictly positive prices \( p \) such that for each individual, endowed with \( e_i+T_i = x_i \), the bundle \( x_i \) is in the Walrasian demand of \( i \):

    \[
        x_i \in D_i(p, e_i+T_i ) \qquad\text{for each } i.
    \]

    \paragraph{Step 1: “Strictly better–than” sets.}

    For each individual \( i \), let

    \[
        \overline{U}^i(x_i) := \{x'_i : x'_i \succ_i x_i\}
    \]

    denote the \textbf{strict} upper contour set at \( x_i \). By continuity and convexity of \( \succsim_i \) this set is convex, and \( x_i \notin \overline{U}^i(x_i) \). Define the set of aggregate improvements

    \[
        \overline{U} (x) = \sum_i \overline{U}^i (x_i)
        :=
        \Bigl\{x' \;\Big|\;  x' = \sum_{i} x'_i \; \text{with} \;  x'_i \in \overline{U}^i(x_i)\text{ for each }i\Bigr\}.
    \]

    So \( \overline{U}(x) \) is the set of all aggregate bundles that can be obtained by letting each individual choose a bundle strictly preferred to her allocation in \( x \). Since it is the sum of convex sets, \( \overline{U} (x) \) is convex. Pareto optimality of \( x \) says that there is no \textbf{feasible} allocation \( x'=(x'_i)_{i} \) with \( x'_i \succ_i x_i \) for all \( i \). Equivalently,

    \[
        \sum_{i} x_i \notin \overline{U} (x).
    \]

    \paragraph{Step 2: A supporting price hyperplane.}
    We have a convex set \( \overline{U}(x) \) and a point \( \sum_i x_i \) outside it. By Theorem \ref{thm:supporting-hyperplane}, there exists a nonzero vector \( p \) such that\footnote{Reversing the inequality does not affect Theorem \ref{thm:supporting-hyperplane}.}

    \[
        p \cdot x' \;\ge\; p \cdot \Bigl(\sum_{i} x_i\Bigr)
        \qquad\text{for all } x' \in \overline{U}(x).
    \]

    In particular, for any profile \( (x'_i)_{i} \) with \( x'_i \in \overline{U}^i(x_i) \) for all \( i \),

    \begin{equation}\label{eq:sftwe}
        \sum_{i} p\cdot x'_i
        =
        p \cdot \Bigl(\sum_{i} x'_i\Bigr)
        \;\ge\;
        p \cdot \Bigl(\sum_{i} x_i\Bigr)
        =
        \sum_{i} p\cdot x_i.
    \end{equation}

    We need to show that prices are strictly positive and that each \(x_i\) is optimal in individual \(i\)'s budget set at prices \(p\) and income \(p\cdot (e_i+T_i)=p\cdot x_i\).

    \paragraph{Step 3: Nonnegativity of supporting prices.}
    Let \(\omega := \sum_{i} x_i\) denote the aggregate bundle at the Pareto optimal allocation. For each good \(k\in\{1,\dots,\ell\}\), let \(\mathbf{e}^k\in\mathbb{R}^\ell\) be the \(k\)-th unit vector.

    Consider the allocation \(\widehat x\) defined by

    \[
        \widehat x_i := x_i + \frac{1}{n}\mathbf{e}^k \qquad\text{for each } i.
    \]

    By monotonicity, \(\widehat x_i \succ_i x_i\) for every \(i\), hence
    \(\sum_i \widehat x_i = \omega + \mathbf{e}^k \in \overline U(x)\).
    Applying Equation \eqref{eq:sftwe} to the profile \((\widehat x_i)_i\) gives

    \[
        p\cdot(\omega+\mathbf{e}^k) \;\ge\; p\cdot \omega,
    \]

    hence \(p\cdot \mathbf{e}^k \ge 0\), i.e. \(p^k\ge 0\). Since \(k\) was arbitrary, \( p \in \mathbb{R}^\ell_{+} \).

    \paragraph{Step 4: Any strictly preferred bundle must cost strictly more.}
    Fix an individual \(j\). We claim that if \(x^{'}_j \succ_j x_j\), then

    \[
        p\cdot x^{'}_j \;>\; p\cdot x_j.
    \]

    \emph{Step 4(a): weak inequality.}
    Suppose \(x^{'}_j \succ_j x_j\). By continuity of \(\succsim_j\), there exists \(\theta\in(0,1)\) close enough to \(0\) such that \((1-\theta)x^{'}_j \succ_j x_j\). Define a profile \(x'\) by

    \[
        x'_j := (1-\theta)x^{'}_j,
        \qquad
        x'_i := x_i + \frac{\theta}{\,n-1\,}x^{'}_j \quad\text{for all } i\neq j.
    \]

    By monotonicity, for each \(i\neq j\) we have \(x'_i \succ_i x_i\), and by construction \(x'_j \succ_j x_j\). Hence \(\sum_i x'_i \in \overline U(x)\), and therefore \eqref{eq:sftwe} implies

    \[
        p\cdot\Bigl(\sum_i x'_i\Bigr) \;\ge\; p\cdot\Bigl(\sum_i x_i\Bigr).
    \]

    But \(\sum_i x'_i = x^{'}_j + \sum_{i\neq j} x_i\), so cancelling \(\sum_{i\neq j}p\cdot x_i\) yields

    \[
        p\cdot x^{'}_j \;\ge\; p\cdot x_j.
    \]

    \emph{Step 4(b): strict inequality.}
    Assume for contradiction that \(p\cdot x^{'}_j = p\cdot x_j\).
    Since \(x^{'}_j \succ_j x_j\) and preferences are continuous, we can pick \(\alpha\in(0,1)\)
    close enough to \(1\) so that \(\alpha x^{'}_j \succ_j x_j\).
    Applying Step 4(a) to \(\alpha x^{'}_j\) gives
    \[
        p\cdot(\alpha x^{'}_j) \;\ge\; p\cdot x_j.
    \]
    Using \(p\cdot x^{'}_j = p\cdot x_j\), this becomes \(\alpha\, p\cdot x_j \ge p\cdot x_j\),
    hence \(p\cdot x_j \le 0\). But \(x_j\in\mathbb{R}^\ell_{++}\) and \(p\in\mathbb{R}^\ell_{+}\) with
    \(p\neq 0\) imply \(p\cdot x_j>0\), a contradiction. Therefore \(p\cdot x^{'}_j > p\cdot x_j\).

    \paragraph{Step 5: Strict positivity and individual optimality.}
    \emph{Strict positivity.} Fix any good \(k\). For any \(\varepsilon>0\),
    monotonicity implies \(x_j+\varepsilon \mathbf{e}^k \succ_j x_j\). By Step 4,
    \[
        p\cdot(x_j+\varepsilon \mathbf{e}^k) \;>\; p\cdot x_j
        \quad\Rightarrow\quad
        \varepsilon p^k>0
        \quad\Rightarrow\quad
        p^k>0.
    \]
    Since \(k\) was arbitrary, \(p\in\mathbb{R}^\ell_{++}\).

    \emph{Optimality.} Fix \(i\). Suppose there exists \(x'_i\in\mathbb{R}^\ell_{+}\) with
    \(p\cdot x'_i \le p\cdot x_i\) and \(x'_i \succ_i x_i\). This contradicts Step 4, which says
    \(x'_i \succ_i x_i \Rightarrow p\cdot x'_i > p\cdot x_i\). Hence no strictly preferred bundle is
    affordable at income \(p\cdot x_i\), so \(x_i\) is optimal in the budget set:
    \[
        x_i \in D_i(p, e_i+T_i).
    \]


    \paragraph{Conclusion.}
    We have found strictly positive prices \(p \in \mathbb{R}^\ell_{++}\) and transfers \((T_i)_i\) such that

    \[
        x_i \in D_i(p, e_i+T_i)\qquad\text{for all } i.
    \]

    Hence \(x\) is a Walrasian equilibrium with transfers: \(x \in R^{WT}(E)\). Since \(x\) was an arbitrary interior Pareto optimal allocation, the theorem follows.
\end{proof}

Let us now take stock of what we have achieved so far. The first fundamental theorem of welfare economics, Theorem \ref{thm:fftwe}, says that every Walrasian equilibrium allocation is Pareto optimal. The second fundamental theorem of welfare economics, Theorem \ref{thm:sftwe}, says that, under stronger assumptions, every interior Pareto optimal allocation can be implemented as a Walrasian equilibrium with transfers. Together, these two theorems establish a strong link between competitive equilibria and efficiency.

\begin{comment}

\begin{itemize}
    \item \textbf{Price-taking:}
    \item \textbf{Complete information};
    \item \textbf{No externalities:}
    \item \textbf{Linear prices:}
\end{itemize}

\end{comment}

\paragraph{Things to read.} This lecture is based on \citet[ch.~17]{varianMicroeconomicAnalysis1992}. For a treatment that includes production, see \citet[pp.~545--550]{mas-colellMicroeconomicTheory1995}.

\section{Exercises}

\begin{exercise}
    If one \textbf{assumes} the existence of a \usename{axn:weqt}, an indirect proof of the second welfare theorem can be given. Prove the following statement. Suppose that all preferences in the economy are locally non-satiated and that \(x^{*}\) is a Pareto optimal allocation. Suppose further that a Walrasian equilibrium exists when endowments are \(e_i = x^{*}_i\) for all \(i\). Then \(x^{*}\) can be supported as a Walrasian equilibrium allocation with transfers. (Hint: if you are stuck, see \citet[p.~329]{varianMicroeconomicAnalysis1992}.)
\end{exercise}


\bibliographystyle{apacite}  % or another  style
\bibliography{references} % .bib file goes in ./bib/