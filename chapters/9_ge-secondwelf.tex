\renewcommand{\thefootnote}{\fnsymbol{footnote}}

\chapter[Second theorem of welfare economics]%
 {Second theorem of welfare economics}%
\label{ch:L9}

% 3) Reset things so later footnotes go back to 1, 2, 3, …
%\setcounter{footnote}{0}
\renewcommand{\thefootnote}{\arabic{footnote}}

%\section{Separating hyperplane}\label{sec:L9-intro}

\begin{theorem}\label{thm:supporting-hyperplane} (\textbf{Separating hyperplane})
    Let \(B \subset \mathbb{R}^L\) be a convex set and let \(x \notin \operatorname{Int} B\)
    (the interior of \(B\)). Then there exists a nonzero vector \(p \in \mathbb{R}^L\) such that
    \[
        p \cdot x \;\ge\; p \cdot y \qquad \text{for every } y \in B.
    \]
    In words: there is a hyperplane through \(x\) that supports \(B\) from one side.
\end{theorem}

\begin{theorem}\label{thm:sftwe} (\textbf{Second fundamental theorem of welfare economics}) If preferences in the economy \( E \) are locally non-satiated, convex, and continuous, and endowments are strictly positive \( e_i \in \mathbb{R}^\ell_{++} \) for all individuals \( i \), then every interior Pareto optimal allocation \( x \in \mathbb{R}^{\ell n}_{++}\) can be supported as a Walrasian equilibrium with transfers. That is,

    \[
        x \quad \text{is Pareto optimal} \quad \Longrightarrow \quad x \in R^{WT}(E).
    \]
\end{theorem}

\begin{proof}
    Say we want to implement the interior Pareto optimal allocation \( x \). Define transfers \( T_i \) by

    \[
        T_i = x_i - e_i \qquad\text{for each } i.
    \]

    Feasibility of \( x \) implies \( \sum_{i} x_i = \sum_{i} e_i \), hence \( \sum_{i} T_i = 0 \), so \( (T_i)_{i} \) is a feasible vector of lump–sum transfers. We have to find strictly positive prices \( p \) such that for each individual, endowed with \( e_i+T_i = x_i \), the bundle \( x_i \) is in the Walrasian demand of \( i \):

    \[
        x_i \in D_i(p, e_i+T_i ) \qquad\text{for each } i.
    \]

    \paragraph{Step 1: “Strictly better–than” sets.}

    For each individual \( i \), let

    \[
        \overline{U}^i(x_i) := \{x'_i : x'_i \succ_i x_i\}
    \]

    denote the \textbf{strict} upper contour set at \( x_i \). By continuity and convexity of \( \succsim_i \) this set is convex, and \( x_i \notin \overline{U}^i(x_i) \). Define the set of aggregate improvements

    \[
        \overline{U} (x) = \sum_i \overline{U}^i (x)
        :=
        \Bigl\{\sum_{i} x'_i \in \mathbb{R}^{\ell}_{++} \;\Big|\; x'_i \in \overline{U}^i(x_i)\text{ for each }i\Bigr\}.
    \]

    So \( \overline{U}(x) \) is the set of all aggregate bundles that can be obtained by letting each individual choose a bundle strictly preferred to her allocation in \( x \). Since it is the sum of convex sets, \( \overline{U} (x) \) is convex. Pareto optimality of \( x \) says that there is no feasible allocation \( x'=(x'_i)_{i} \) with \( x'_i \succ_i x_i \) for all \( i \). Equivalently,

    \[
        \sum_{i} x_i \notin \overline{U} (x).
    \]

    \paragraph{Step 2: A supporting price hyperplane.}
    We have a convex set \( \overline{U}(x) \) and a point \( \sum_i x_i \) outside it. By the supporting hyperplane theorem, there exists a nonzero vector \( p \in \mathbb{R}^\ell_{+} \) such that

    \[
        p \cdot x' \;\ge\; p \cdot \Bigl(\sum_{i} x_i\Bigr)
        \qquad\text{for all } x' \in \overline{U}(x).
    \]

    In particular, for any profile \( (x'_i)_{i} \) with \( x'_i \in \overline{U}^i(x_i) \) for all \( i \),

    \begin{equation}\label{eq:sftwe}
        \sum_{i} p\cdot x'_i
        =
        p \cdot \Bigl(\sum_{i} x'_i\Bigr)
        \;\ge\;
        p \cdot \Bigl(\sum_{i} x_i\Bigr)
        =
        \sum_{i} p\cdot x_i.
    \end{equation}

    \paragraph{Step 3: Prices must be strictly positive.}
    We now show that prices are strictly positive, i.e., \( p \in \mathbb{R}^\ell_{++} \).

    \emph{(i) No component of \( p \) can be negative.} Suppose \( p_\ell < 0 \) for some \( \ell \). Pick any individual \( i \). By monotonicity, for any small \( \varepsilon>0 \) the bundle

    \[
        x'_i := x_i + \varepsilon e_\ell
    \]

    satisfies \( x'_i \succ_i x_i \), hence \( x'_i \in \overline{U}^i(x_i) \), while all
    other individuals \( j\neq i \) keep \( x'_j = x_j \). Thus \( x' := \sum_i x'_i \in \overline{U}(x) \). By the supporting hyperplane inequality from Step~2,

    \[
        p\cdot x' \;\ge\; p\cdot \sum_i x_i.
    \]

    But

    \[
        p\cdot x'
        =
        p\cdot \sum_i x_i + \varepsilon p_\ell
        <
        p\cdot \sum_i x_i,
    \]

    a contradiction. Hence \( p_\ell \ge 0 \) for each \( \ell \), so \( p \in \mathbb{R}^\ell_{+} \).

    \emph{(ii) No component of \( p \) can be zero.} Suppose \( p_\ell = 0 \) for some \( \ell \). Again choose any individual \( i \). Because \(x_i \in \mathbb{R}^\ell_{++}\), there is some small \( \varepsilon>0 \) such that \(x_i - \varepsilon \mathbf{1} \in \mathbb{R}^\ell_{+}\), where \(\mathbf{1}\) is the vector of all ones. Consider bundles \(y_i\) of the form

    \[
        y^i_\ell > x^i_\ell,
        \qquad
        y^i_{k} = x^i_{k} - \varepsilon
        \ \text{ for all } k \neq \ell.
    \]

    By continuity and monotonicity of preferences, we can choose \(y_i\) of this form (with \(\varepsilon\) small and \(y^i_\ell\) large enough) so that \( y_i \succ_i x_i \), hence \(y_i \in \overline{U}^i(x_i)\). For all \( j\neq i \), set \( y_j := x_j \). Then \( y := \sum_j y_j \in \overline{U}(x) \), so Step~2 gives

    \[
        p\cdot y \;\ge\; p\cdot \sum_j x_j.
    \]

    But

    \[
        p\cdot y
        =
        p\cdot \sum_j x_j
        +
        \sum_{k\neq \ell} (y^i_k - x^i_k) p_k
        +
        (y^i_\ell - x^i_\ell)p_\ell.
    \]

    Here \(y^i_k - x^i_k = -\varepsilon <0\) for all \(k\neq \ell\) and \(p_k \ge 0\), while \(p_\ell = 0\). Since \(p\neq 0\), at least one of the \(p_k\) with \(k\neq \ell\) is strictly positive, so

    \[
        \sum_{k\neq \ell} (y^i_k - x^i_k) p_k < 0,
    \]

    and hence \(p\cdot y < p\cdot \sum_j x_j\), a contradiction. Therefore no component of \(p\) can be zero either, and we conclude that \(p \in \mathbb{R}^\ell_{++}\).

    \paragraph{Step 4: Individual optimality.} Fix an individual \(i\). We show that \(x_i\) is optimal in her budget set. Suppose, by contradiction, that there exists \(x'_i \in \overline{U}^i(x_i)\) with \(p\cdot x'_i \le p\cdot x_i\). Keep all other individuals at their original bundles: \(x'_j := x_j\) for every \(j\neq i\). Then the profile \((x'_k)_k\) satisfies \(x'_k \in \overline{U}^k(x_k)\) for all \(k\), so \(\sum_k x'_k \in \overline{U}(x)\). By Equation \eqref{eq:sftwe} we must have

    \[
        \sum_k p\cdot x'_k \;\ge\; \sum_k p\cdot x_k.
    \]

    On the other hand,

    \[
        \sum_k p\cdot x'_k
        =
        p\cdot x'_i + \sum_{j\neq i} p\cdot x_j
        \le
        p\cdot x_i + \sum_{j\neq i} p\cdot x_j
        =
        \sum_k p\cdot x_k,
    \]

    with strict inequality if \(p\cdot x'_i < p\cdot x_i\). This contradicts \eqref{eq:sftwe}. Hence no such \(x'_i\) exists, and \(x_i\) is optimal at prices \(p\), that is,

    \[
        x_i \in D_i(p, e_i+T_i).
    \]

    \paragraph{Conclusion.}
    We have found prices \(p \in \mathbb{R}^\ell_{++}\) and transfers \((T_i)_i\) such that
    \[
        x_i \in D_i(p, e_i+T_i)\qquad\text{for all } i.
    \]
    Hence \(x\) is a Walrasian equilibrium with transfers: \(x \in R^{WT}(E)\). Since \(x\) was an arbitrary interior Pareto optimal allocation, the theorem follows.
\end{proof}

\paragraph{Things to read.} This lecture is based on \citet[pp. 545-550]{mas-colellMicroeconomicTheory1995}.

\section{Exercises}

hey.

\bibliographystyle{apacite}  % or another  style
\bibliography{references} % .bib file goes in ./bib/