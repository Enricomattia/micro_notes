\renewcommand{\thefootnote}{\fnsymbol{footnote}}

\chapter[General Equilibrium]%
 {General Equilibrium}%
\label{ch:L6}

% 3) Reset things so later footnotes go back to 1, 2, 3, …
%\setcounter{footnote}{0}
\renewcommand{\thefootnote}{\arabic{footnote}}

\section{Exchange economies}\label{sec:L6-intro}

There is a set of individuals \( I = \{1,\dots,n\} \) and a set of consumption
bundles \( \mathbb{R}^{\ell}_+ \). Each individual has preferences \( \succsim_i \) on \( \mathbb{R}^{\ell}_+ \). Each agent has an endowment \( e_i \in \mathbb{R}^{\ell}_+ \) and therefore there is a total endowment of

\[
	\sum_{i \in I} e_i = \bar{e}.
\]

An \textbf{economy} is a profile \( E = ((\succsim_i, e_i)_{i \in I}) \). A \textbf{feasible} allocation for \( E \) is \( x = (x_i)_{i \in I} \) such that

\[
	\sum_{i \in I} x_i \le \bar{e}.
\]

\begin{definition}
	The budget set given endowment \( e_i \) and prices \( p \in \mathbb{R}^{\ell}_{++} \), is

	\[
		B(e_i,p)
		= \left\{
		x \in \mathbb{R}^{\ell}_+ \;\middle|\; p \cdot x \le p \cdot e_i
		\right\}.
	\]
\end{definition}

Notice that the budget set is always compact and convex.

\begin{definition}
	The \textbf{Walrasian demand} of \( i \), given endowment \( e_i \) and prices \( p \), is
	\[
		D_i(e_i,p)
		= \left\{
		x \in B(e_i,p) \;\middle|\;
		x \succsim_i y \ \forall y \in B(e_i,p)
		\right\}
	\]
\end{definition}

Let \( F(E) \) be the set of feasible allocations. An \textbf{allocation rule} \( R \) maps an economy \( E \) to a subset of feasible allocations \( R(E) \subseteq F(E) \).

\begin{definition}
	An allocation rule \(R^{W}\) is \textbf{Walrasian} if for all economies \(E\)
	there exist prices \(p \in \mathbb{R}^\ell_{++}\) such that
	\[
		R^{W}(E)
		=
		\left\{
		x \in F(E) \;\middle|\;
		\forall i \in I,\ x_i \in D_i(e_i, p)
		\right\}.
	\]
\end{definition}

\begin{definition}
	An allocation rule \(R^{EW}\) is \textbf{Egalitarian Walrasian} if for all
	economies \(E\) there exist prices \(p \in \mathbb{R}^\ell_{++}\) such that
	\[
		R^{EW}(E)
		=
		\left\{
		x \in F(E) \;\middle|\;
		\forall i \in I,\
		x_i \in D_i\!\left(\frac{\bar e}{|I|},\, p\right)
		\right\}.
	\]
\end{definition}

\begin{definition}
	An allocation \( x \) is \textbf{Pareto optimal} if there is no other feasible allocation \( x' \) with \( x'_i \succsim_i x_i \) for all \( i \) and \( x'_j \succ_j x_j \) for some \( j \).
\end{definition}

\begin{definition}
	An allocation rule \( R \) satisfies \textbf{weak Pareto} if for all economies \( E \), \( R(E) \) contains only Pareto optimal allocations.
\end{definition}

\begin{definition}
	An allocation rule \(R^{WT}\) is a \textbf{Walrasian equilibrium with transfers} if for all economies \(E\) there exist strictly positive prices \(p \in \mathbb{R}^\ell_{++}\) and transfers \((T_i)_{i\in I}\) satisfying
	\[
		\sum_{i\in I} T_i = 0
		\quad\text{and}\quad
		e_i + T_i \in \mathbb{R}^\ell_{+}\ \text{for all } i,
	\]
	such that
	\[
		R^{WT}(E)
		=
		\left\{
		x \in F(E) \;\middle|\;
		\forall i \in I,\ x_i \in D_i(e_i + T_i, p)
		\right\}.
	\]
\end{definition}

\paragraph{Things to read.} It might be useful for you to review (or study, if you never encountered these topics before), \citet[pp. 51-70, 76-84]{hildenbrandIntroductionEquilibriumAnalysis1976}. If you want (and you \textquote{should want}) to go deeper, study study \citet[pp. 17–23, 40–56]{mas-colellMicroeconomicTheory1995}. \cite{arrowSocialChoiceIndividual2012}.

\section{Exercises}

hey.

\bibliographystyle{apacite}  % or another  style
\bibliography{references} % .bib file goes in ./bib/